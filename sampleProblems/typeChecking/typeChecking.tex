\documentclass[addpoints]{exam}

%\printanswers
\noprintanswers

\usepackage{amsmath,bigstrut,minted,url,graphicx,tikz,marvosym,rotating}
\usepackage{ebproof}

\pagestyle{headandfoot}
\runningheadrule
\firstpageheader{}{}{Page \thepage\ of \numpages}
\runningheader{CS 321}{Sample Problems}{Page \thepage\ of \numpages}
\firstpagefooter{}{}{}
\runningfooter{}{}{}
              
\newcommand{\EJDG}[3]{#1 \vdash #2 \Rightarrow #3}
\newcommand{\TJDG}[3]{#1 \vdash #2 : #3}
\renewcommand{\L}[2]{\lambda #1.#2}

\begin{document}

\begin{center}
{\Large \textbf{
    Ozyegin University\\
    CS 321 Programming Languages\\
    Sample Problems on Type Checking
}}
\end{center}

\section*{Reference}
Typing rules of the Deve language are given below.

\input{../../../lectures/typeChecking/typingRules.tex}

\newpage
The \mintinline{ocaml}{typeOf} function is given below.

{\small
\begin{minted}[mathescape,frame=single]{ocaml}
type typ = IntTy
         | BoolTy
         | PairTy of typ * typ
         | FunTy of typ * typ

(* typeOf: exp -> (string * typ) list -> typ *)
let rec typeOf e tyEnv =
  match e with
  | CstI i -> IntTy
  | CstB b -> BoolTy
  | Var x -> lookup x tyEnv
  | Binary(op, e1, e2)  ->
     let t1 = typeOf e1 tyEnv in
     let t2 = typeOf e2 tyEnv in
     (match op, t1, t2 with
      | "+", IntTy, IntTy -> IntTy
      | "-", IntTy, IntTy -> IntTy
      | "*", IntTy, IntTy -> IntTy
      | "/", IntTy, IntTy -> IntTy
      | "<", IntTy, IntTy -> BoolTy
      | "<=", IntTy, IntTy -> BoolTy
      | ",", _, _ -> PairTy(t1, t2)
      | _ -> failwith ("Bad use of the binary operator: " ^ op)
     )
  | LetIn(x, e1, e2) ->
     let t = typeOf e1 tyEnv
     in let tyEnv' = (x, t)::tyEnv
        in typeOf e2 tyEnv'
  | LetRec(f, (x,t1), retTy, e1, e2) ->
     let tBody = typeOf e1 ((f, FunTy(t1,retTy))::(x,t1)::tyEnv)
     in if tBody = retTy then
          typeOf e2 ((f, FunTy(t1,retTy))::tyEnv)
        else failwith "Return type of the rec. function should agree with the type of the bofy."
  | If(e1, e2, e3) -> (match typeOf e1 tyEnv with
                       | BoolTy -> let t2 = typeOf e2 tyEnv in
                                   let t3 = typeOf e3 tyEnv in
                                   if t2 = t3 then t2
                                   else failwith "Branch types of an if-then-else must agree." 
                       | _ -> failwith "Condition should be a bool.")
  | MatchPair(e1, x, y, e2) ->
     (match typeOf e1 tyEnv with
      | PairTy(t1, t2) -> typeOf e2 ((x,t1)::(y,t2)::tyEnv)
      | _ -> failwith "Pair pattern matching works on pair values only (obviously)!"
     )
  | Fun((x, t), e) ->
     let tBody = typeOf e ((x,t)::tyEnv)
     in FunTy(t, tBody)
  | App(e1, e2) ->
     (match typeOf e1 tyEnv with
      | FunTy(t2, t1) ->
         if t2 = typeOf e2 tyEnv then t1
         else failwith "Function parameter type should agree with the argument type."
      | _ -> failwith "Application wants to see a function!"
     )  
\end{minted}
}

\newpage
\section*{Questions}
\begin{questions}
  \question
  For each of the program points below,
  write down the \emph{type environment}. Assume that we start
  with the empty environment. 

  \begin{minipage}{0.35\textwidth}
  \begin{parts}
    \part
    \begin{minted}{ocaml}
let x = 9 in
(* program point 1 *)
let f y = x * y in
(* program point 2 *)
let x = 4 in
(* program point 3 *)
let y = 7 in
(* program point 4 *)
f x
    \end{minted}

    \part
    \begin{minted}{ocaml}
let x = 9 in
(* program point 1 *)
let y = let x = 13 in
    (* program point 2 *)
        x + 2 in
(* program point 3 *)
y + x
    \end{minted}

    \part
    \begin{minted}{ocaml}
let add x y = x + y in
(* program point 1 *)
let foo = add 10 in
(* program point 2 *)
let baz = foo 20 in
(* program point 3 *)
baz
    \end{minted}
  \end{parts}
  \end{minipage}%
  \begin{minipage}{0.65\textwidth}
    \begin{solutionbox}{3.5cm}
      \begin{itemize}
      \item 1: $[\texttt{x}\mapsto int]$
      \item 2: $[\texttt{f}\mapsto (int\to int), \texttt{x}\mapsto int]$
      \item 3: $[\texttt{x}\mapsto int, \texttt{f}\mapsto (int\to int), \texttt{x}\mapsto int]$
      \item 4: $[\texttt{y}\mapsto int, \texttt{x}\mapsto int, \texttt{f}\mapsto (int\to int), \texttt{x}\mapsto int]$
      \end{itemize}
    \end{solutionbox}

    \begin{solutionbox}{3cm}
      \begin{itemize}
      \item 1: $[\texttt{x}\mapsto int]$
      \item 2: $[\texttt{x}\mapsto int, \texttt{x}\mapsto int]$
      \item 3: $[\texttt{y}\mapsto int, \texttt{x}\mapsto int]$
      \end{itemize}
    \end{solutionbox}

    \begin{solutionbox}{3cm}
      \begin{itemize}
      \item 1: $[\texttt{add}\mapsto (int \to int \to int)]$
      \item 2: $[\texttt{foo}\mapsto (int \to int), \texttt{add}\mapsto (int \to int \to int)]$
      \item 3: $[\texttt{baz} \mapsto int, \texttt{foo}\mapsto (int \to int), \texttt{add}\mapsto (int \to int \to int)]$
      \end{itemize}
    \end{solutionbox}
  \end{minipage}

  
  \question
  Suppose we had ``min'' and ``max'' as binary operators.
  Define typing rules for them and
  also show how the \mintinline{ocaml}{typeOf} function would be implemented.

  \begin{solutionbox}{4cm}
    \[
    \frac{\TJDG{\rho}{e_1}{\mathrm{int}} \qquad \TJDG{\rho}{e_2}{\mathrm{int}}}
         {\TJDG{\rho}{\texttt{min($e_1$,$e_2$)}}{\mathrm{int}}}
         \qquad \textrm{(and similarly for \texttt{max})}
    \]

    For the implementation, add a new Binary operator case for each.

    \begin{minted}{ocaml}
      | "min", IntTy, IntTy -> BoolTy
      | "max", IntTy, IntTy -> BoolTy
    \end{minted}
  \end{solutionbox}
  
  \question
  Suppose we had ``='' as a binary operator for equality checking.
  Define typing rules for this operator and
  also show how the \mintinline{ocaml}{typeOf} function would be implemented.
  ``='' works for between any pair of values as long as they have the same type.
  E.g. These are fine: \texttt{4 = 6}, \texttt{(4<5) = true}, \texttt{(4,5) = (3+1,10/2)}

  \begin{solutionbox}{5cm}
    \[
    \frac{\TJDG{\rho}{e_1}{\tau} \qquad \TJDG{\rho}{e_2}{\tau}}
         {\TJDG{\rho}{\texttt{$e_1$=$e_2$}}{\mathrm{bool}}}
    \]

    For the implementation, add a new Binary operator case.

    \begin{minted}{ocaml}
      | "=", _, _ when t1 = t2 -> BoolTy
    \end{minted}

    Or:

    \begin{minted}{ocaml}
      | "=", _, _ -> if t1 = t2 then BoolTy
                     else failwith "Bad use of = operator."
    \end{minted}
  \end{solutionbox}

  \question
  Suppose we had unary operators in the language,
  represented with the \mintinline{ocaml}{Unary of string * exp}
  constructor.
  Define typing rules for the ``fst'' and ``snd'' unary operators and
  also show how the \mintinline{ocaml}{typeOf} function would be implemented.

  \begin{solutionbox}{7cm}
    \[
    \frac{\TJDG{\rho}{e}{(\tau_1\times\tau_2)}}
         {\TJDG{\rho}{\texttt{fst($e$)}}{\tau_1}}
    \qquad
    \frac{\TJDG{\rho}{e}{(\tau_1\times\tau_2)}}
         {\TJDG{\rho}{\texttt{snd($e$)}}{\tau_2}}
    \]

    For the implementation, add a new case.

    \begin{minted}{ocaml}
let rec typeOf e tyEnv =
  match e with
  ...
  | Unary(op, e)  ->
     let t = typeOf e tyEnv in
     (match op, t1, t2 with
      | "fst", PairTy(t1, t2) -> t1
      | "snd", PairTy(t1, t2) -> t2
      | _ -> failwith ("Bad use of the unary operator: " ^ op)
     )
    \end{minted}
  \end{solutionbox}

  \question
  Using the Deve typing rules,
  show the type derivation tree for the type judgment
  given below.
  \[
  \small
  \TJDG{[\,]}{\texttt{let x = 1 in x < 2}}{bool} 
  \]

  \begin{solutionbox}{3cm}
    \[
    \begin{prooftree}
        \infer0[(1)]{\TJDG{[\,]}{\texttt{1}}{int}}
            \hypo{[\texttt{x}\mapsto int](x) = int}
          \infer1[(3)]{\TJDG{[\texttt{x}\mapsto int]}{\texttt{x}}{int}}
          \infer0[(1)]{\TJDG{[\texttt{x}\mapsto int]}{\texttt{2}}{int}}
        \infer2[(5)]{\TJDG{[\texttt{x}\mapsto int]}{\texttt{x < 2}}{bool}}
      \infer2[(8)]{\TJDG{[\,]}{\texttt{let x = 1 in x < 2}}{bool}}
    \end{prooftree}
    \]
  \end{solutionbox}

  \question
  Using the Deve typing rules,
  show the type derivation tree for the type judgment
  given below.
  \[
  \small
  \TJDG{[\,]}{\texttt{let z = 1<2 in if z then 3 else 4}}{int} 
  \]

  \begin{solutionbox}{3cm}
    \[
    \small
    \begin{prooftree}
          \infer0[(1)]{\TJDG{[\,]}{\texttt{1}}{int}}
          \infer0[(1)]{\TJDG{[\,]}{\texttt{2}}{int}}
        \infer2[(5)]{\TJDG{[\,]}{\texttt{1 < 2}}{bool}}
            \hypo{[\texttt{z}\mapsto bool](\texttt{z}) = bool}  
          \infer1[(3)]{\TJDG{[\texttt{z}\mapsto bool]}{\texttt{z}}{bool}}
          \infer0[(1)]{\TJDG{[\texttt{z}\mapsto bool]}{\texttt{3}}{int}}
          \infer0[(1)]{\TJDG{[\texttt{z}\mapsto bool]}{\texttt{4}}{int}}
        \infer3[(7)]{\TJDG{[\texttt{z}\mapsto bool]}{\texttt{if z then 3 else 4}}{int}}
      \infer2[(8)]{\TJDG{[\,]}{\texttt{let z = 1<2 in if z then 3 else 4}}{int}}
    \end{prooftree}
    \]
  \end{solutionbox}

  \question
  Each of the following expressions has a problem that prevents it from being 
  accepted by the Deve type system. In other words, it is impossible 
  to construct a type derivation tree. Explain at which rule your attempt to
  build a tree would fail, and why.

  \begin{itemize}
  \item $\TJDG{[\texttt{y}\mapsto bool]}{\texttt{y < 42}}{bool}$
    \begin{solutionbox}{3cm}
      Fails when attempting to use rule 3:
      \[
      \begin{prooftree}
          \hypo{\textrm{\normalsize\Frowny}}
          \infer1[(3)]{\TJDG{[\texttt{y}\mapsto bool]}{\texttt{y}}{int}}
          \infer0[(1)]{\TJDG{[\texttt{y}\mapsto bool]}{\texttt{42}}{int}}
        \infer2[(5)]{\TJDG{[\texttt{y}\mapsto bool]}{\texttt{y < 42}}{bool}}
      \end{prooftree}
      \]
    \end{solutionbox}

  \item $\TJDG{[\,]}{\texttt{let x = 17 in x 25}}{int}$
    \begin{solutionbox}{3.5cm}
      \[
      \begin{prooftree}
          \infer0[(1)]{\TJDG{[\,]}{\texttt{17}}{int}}
             \hypo{\textrm{\normalsize\Frowny}}
            \infer1[(3)]{\TJDG{[\texttt{x}\mapsto int]}{\texttt{x}}{(int \to int)}}
            \infer0[(1)]{\TJDG{[\texttt{x}\mapsto int]}{\texttt{25}}{int}}
          \infer2[(10)]{\TJDG{[\texttt{x}\mapsto int]}{\texttt{x 25}}{int}}
        \infer2[(8)]{\TJDG{[\,]}{\texttt{let x = 17 in x 25}}{int}}
      \end{prooftree}
      \]
    \end{solutionbox}

  \item $\TJDG{[\texttt{x}\mapsto int]}{\texttt{if x < 0 then 54 else false}}{int}$
    \begin{solutionbox}{3cm}
      \[\small
      \begin{prooftree}
            \hypo{[\texttt{x}\mapsto int](\texttt{x}) = int}
            \infer1[(3)]{\TJDG{[\texttt{x}\mapsto int]}{\texttt{x}}{int}}
            \infer0[(1)]{\TJDG{[\texttt{x}\mapsto int]}{\texttt{0}}{int}}
          \infer2[(5)]{\TJDG{[\texttt{x}\mapsto int]}{\texttt{x < 0}}{bool}}
          \infer0[(1)]{\TJDG{[\texttt{x}\mapsto int]}{\texttt{54}}{int}}
            \hypo{\textrm{\normalsize\Frowny}}
          \infer1[(2)]{\TJDG{[\texttt{x}\mapsto int]}{\texttt{false}}{int}}
        \infer3[(7)]{\TJDG{[\texttt{x}\mapsto int]}{\texttt{if x < 0 then 54 else false}}{int}}
      \end{prooftree}
      \]
    \end{solutionbox}
    
  \end{itemize}


  \question
  Using the Deve typing rules,
  show the type derivation tree for the type judgment
  given below.

  \[
  \TJDG{[\,]}{\texttt{let x = 3+5 in if x<0 then (fun n -> n*2) else (fun z -> z-x)}}{int \to int}
  \]

  \begin{solutionbox}{1.5cm}
    This was done in the lecture. Check your notes, or buy coffee for a friend who takes notes.
  \end{solutionbox}

  \newpage
  \question
  Using the Deve typing rules,
  show the type derivation tree for the type judgment
  given below.
  \[
  \small
  \TJDG{[\,]}{\texttt{let rec fib (n:int) :int = if n<2 then n else fib(n-1) + fib(n-2) in fib 42}}{int} 
  \]

  \begin{solutionbox}{20cm}
    In the solution below, $\rho_1$ stands for the following environment:
    $[\texttt{fib}\mapsto (int \to int), \texttt{n} \mapsto int]$. Also,
    $\rho_2$ stands for the following environment:
    $[\texttt{fib}\mapsto (int \to int)]$.

    \vspace{1cm}
  \begin{sideways}
    \begin{minipage}{\textwidth}
      \[
      \scriptsize
      \begin{prooftree}
                \hypo{\rho_1(\texttt{n}) = int}
              \infer1[(3)]{\TJDG{\rho_1}{\texttt{n}}{int}}
              \infer0[(1)]{\TJDG{\rho_1}{\texttt{2}}{int}}
            \infer2[(5)]{\TJDG{\rho_1}{\texttt{n<2}}{bool}}
              \hypo{\rho_1(\texttt{n}) = int}
            \infer1[(3)]{\TJDG{\rho_1}{\texttt{n}}{int}}
            \hypo{\textrm{\Large $\mathcal{C}$}}
          \infer3[(7)]{\TJDG{\rho_1}{\texttt{if n<2 then n else fib(n-1) + fib(n-2)}}{int}}
              \hypo{\rho_2(fib) = (int \to int)}
            \infer1[(3)]{\TJDG{\rho_2}{\texttt{fib}}{(int \to int)}}
            \infer0[(1)]{\TJDG{\rho_2}{\texttt{42}}{int}}
          \infer2[(10)]{\TJDG{\rho_2}{\texttt{fib 42}}{int}}
        \infer2[(12)]{\TJDG{[\,]}{\texttt{let rec fib (n:int) :int = if n<2 then n else fib(n-1) + fib(n-2) in fib 42}}{int}}
      \end{prooftree}
      \]

      \vspace{2cm}
      And tree {\Large $\mathcal{C}$} is:
      
      \[
      \scriptsize
      \begin{prooftree}
                    \hypo{\rho_1(\texttt{fib}) = (int\to int)}
                \infer1[(3)]{\TJDG{\rho_1}{\texttt{fib}}{(int\to int)}}
                    \hypo{\rho_1(\texttt{n}) = int}
                  \infer1[(3)]{\TJDG{\rho_1}{\texttt{n}}{int}}
                  \infer0[(1)]{\TJDG{\rho_1}{\texttt{1}}{int}}
                \infer2[(4)]{\TJDG{\rho_1}{\texttt{n-1}}{int}}
              \infer2[(10)]{\TJDG{\rho_1}{\texttt{fib(n-1)}}{int}}
                  \hypo{\rho_1(\texttt{fib}) = (int\to int)}
                \infer1[(3)]{\TJDG{\rho_1}{\texttt{fib}}{(int\to int)}}
                    \hypo{\rho_1(\texttt{n}) = int}
                  \infer1[(3)]{\TJDG{\rho_1}{\texttt{n}}{int}}
                  \infer0[(1)]{\TJDG{\rho_1}{\texttt{2}}{int}}
                \infer2[(4)]{\TJDG{\rho_1}{\texttt{n-2}}{int}}
              \infer2[(10)]{\TJDG{\rho_1}{\texttt{fib(n-2)}}{int}}
            \infer2[(4)]{\TJDG{\rho_1}{\texttt{fib(n-1) + fib(n-2)}}{int}}
      \end{prooftree}
      \]
    \end{minipage}
  \end{sideways}
  \end{solutionbox}

  \newpage
  \question
  What are the types of the following OCaml
  expressions? Give types that are as general as possible.
  You may use Greek letters (e.g. $\alpha, \beta, \gamma, \delta$ etc.)
  or quoted letters (e.g. \texttt{'a,'b,'c,'d} etc.)
  for polymorphic types.

  \begin{parts}
    \part
    \begin{minted}{ocaml}
      let q3 f = f(f(f(1)))
    \end{minted}
    \begin{solutionbox}{1cm}
      \mintinline{ocaml}{(int -> int) -> int}
    \end{solutionbox}
    
    \part
    \begin{minted}{ocaml}
      let q4 f n = f(f(f(n)))
    \end{minted}
    \begin{solutionbox}{1cm}
      \mintinline{ocaml}{('a -> 'a) -> 'a -> 'a}
    \end{solutionbox}
    
    \part
    \begin{minted}{ocaml}
      let q6 p1 p2 = (snd p2, fst p2, snd p1, fst p1)
    \end{minted}
    \begin{solutionbox}{1cm}
      \mintinline{ocaml}{'a * 'b -> 'c * 'd -> 'd * 'c * 'b * 'a}
    \end{solutionbox}
    
    \part
    \begin{minted}{ocaml}
      let rec graph f lst =
        match lst with
        | [] -> []
        | (x::xs) -> (x,f x) :: graph f xs
    \end{minted}
    \begin{solutionbox}{1cm}
      \mintinline{ocaml}{('a -> 'b) -> 'a list -> ('a * 'b) list}
    \end{solutionbox}
    
    \part
    \begin{minted}{ocaml}
      let rec fold f a lst = 
        match lst with
        | [] -> a
        | x::xs -> fold f (f a x) xs
    \end{minted}
    \begin{solutionbox}{1cm}
      \mintinline{ocaml}{('a -> 'b -> 'a) -> 'a -> 'b list -> 'a}
    \end{solutionbox}
    
    \part
    \begin{minted}{ocaml}
      type 'a tree = Leaf of 'a 
                   | Node of ('a * 'a tree * 'a tree)
                 
      let rec flatten t =
        match t with 
        | Leaf n -> [n]
        | Node(n, t1, t2) -> flatten t1 @ (n::flatten t2);;
    \end{minted}
    \begin{solutionbox}{1cm}
      \mintinline{ocaml}{'a tree -> 'a list}
    \end{solutionbox}
    
    \part
    \begin{minted}{ocaml}
      let p = (34, true);;
    \end{minted}
    \begin{solutionbox}{1cm}
      \mintinline{ocaml}{int * bool}
    \end{solutionbox}
    
    \part
    \begin{minted}{ocaml}
      let f x = (x, (x+5, x > 0));;  
    \end{minted}
    \begin{solutionbox}{1cm}
      \mintinline{ocaml}{int -> int * (int * bool)}
    \end{solutionbox}
    
    \part
    \begin{minted}{ocaml}
      let f x y = (y, x);;
    \end{minted}
    \begin{solutionbox}{1cm}
      \mintinline{ocaml}{'a -> 'b -> 'b * 'a}
    \end{solutionbox}
    
    \part
    \begin{minted}{ocaml}
      let f (x,y) = (y, x);;
    \end{minted}
    \begin{solutionbox}{1cm}
      \mintinline{ocaml}{'a * 'b -> 'b * 'a}
    \end{solutionbox}
    
    \part
    \begin{minted}{ocaml}
      let f x = List.map (fun y -> y*y) x;;
    \end{minted}
    \begin{solutionbox}{1cm}
      \mintinline{ocaml}{int list -> int list}
    \end{solutionbox}
    
    \part
    \begin{minted}{ocaml}
      let f x g b = List.fold_left g b x;;        
    \end{minted}
    \begin{solutionbox}{1cm}
      \mintinline{ocaml}{'a list -> ('b -> 'a -> 'b) -> 'b -> 'b}
    \end{solutionbox}
    
    \part
    \begin{minted}{ocaml}
      let rec f p =       
        match p with 
        | [] -> []
        | x::xs -> (x+x)::f xs;; 
    \end{minted}
    \begin{solutionbox}{1cm}
      \mintinline{ocaml}{int list -> int list}
    \end{solutionbox}
    
    \part
    \begin{minted}{ocaml}
      let f = let max n m = if n - m > 0 then n else m
              in max 10;;
    \end{minted}
    \begin{solutionbox}{1cm}
      \mintinline{ocaml}{int -> int}
    \end{solutionbox}
    
    \part
    \begin{minted}{ocaml}
      let f g x = g(g(g(x)));;     
    \end{minted}
    \begin{solutionbox}{1cm}
      \mintinline{ocaml}{('a -> 'a) -> 'a -> 'a}
    \end{solutionbox}
    
    \part
    \begin{minted}{ocaml}
      let apply f x y = f x y;;
    \end{minted}
    \begin{solutionbox}{1cm}
      \mintinline{ocaml}{('a -> 'b -> 'c) -> 'a -> 'b -> 'c}
    \end{solutionbox}
    
    \part
    \begin{minted}{ocaml}
      let compose f g x = f(g(x));;
    \end{minted}
    \begin{solutionbox}{1cm}
      \mintinline{ocaml}{('a -> 'b) -> ('c -> 'a) -> 'c -> 'b}
    \end{solutionbox}
    
    \part
    \begin{minted}{ocaml}
      let rec g f a lst =
        match lst with
        | [] -> a
        | x::xs -> g f (f a x) xs;;
    \end{minted}
    \begin{solutionbox}{1cm}
      \mintinline{ocaml}{('a -> 'b -> 'a) -> 'a -> 'b list -> 'a}
    \end{solutionbox}
    
    \part
    \begin{minted}{ocaml}
      let f x = if x > 0 then Some x else None;;
    \end{minted}
    \begin{solutionbox}{1cm}
      \mintinline{ocaml}{int -> int option}
    \end{solutionbox}
    
    \part
    \begin{minted}{ocaml}
      let rec last p lst =                             
        match lst with                                 
        | [] -> None                                      
        | x::xs -> (match last p xs with                
                    | None -> if p x then Some x else None
                    | Some y -> Some y);;
    \end{minted}
    \begin{solutionbox}{1cm}
      \mintinline{ocaml}{('a -> bool) -> 'a list -> 'a option}
    \end{solutionbox}
    
    \part
    \begin{minted}{ocaml}
      let rec f lst a =                     
        match lst with                      
        | [] -> a                           
        | x::xs -> f xs (x::a);;            
    \end{minted}
    \begin{solutionbox}{1cm}
      \mintinline{ocaml}{'a list -> 'a list -> 'a list}
    \end{solutionbox}
    
  \end{parts}
  
  \question
  Give the most general (polymorphic) types for the following functions.
  If the function is not typable, write ERROR and explain the problem.

  \begin{parts}
    \part
    \begin{minted}{ocaml}
      let f id = (id 5, id true)
    \end{minted}
    \begin{solutionbox}{1cm}
      ERROR. Function parameter cannot be polymorphic.
    \end{solutionbox}
    
    \part
    \begin{minted}{ocaml}
      let rec gee f xs =
        match xs with
        | [] -> []
        | y::ys -> (y, f y)::(gee f ys)
    \end{minted}
    \begin{solutionbox}{1cm}
      \mintinline{ocaml}{('a -> 'b) -> 'a list -> ('a * 'b) list} 
    \end{solutionbox}
    
    \part
    \begin{minted}{ocaml}
      let rec f n = f (n+1)
    \end{minted}
    \begin{solutionbox}{1cm}
      \mintinline{ocaml}{int -> 'a}
    \end{solutionbox}
  \end{parts}
  
\end{questions}
\end{document}
