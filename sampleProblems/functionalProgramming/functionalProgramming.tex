\documentclass[addpoints]{exam}

%\printanswers
\noprintanswers

\usepackage{amsmath,bigstrut,minted,url}

\pagestyle{headandfoot}
\runningheadrule
\firstpageheader{}{}{Page \thepage\ of \numpages}
\runningheader{CS 321}{Sample Problems}{Page \thepage\ of \numpages}
\firstpagefooter{}{}{}
\runningfooter{}{}{}
              
\begin{document}

\begin{center}
{\Large \textbf{
    Ozyegin University\\
    CS 321 Programming Languages\\
    Sample Problems on Functional Programming
}}
\end{center}

\begin{questions}
  \question 
  Write a function \texttt{stringy : string list -> (string * int) list}
  that associates each string in its input with the length of the string.
  You may use \texttt{String.length} to find the length of a string.
  \begin{minted}{ocaml}
    # stringy ["a"; "bbb"; "cc"; "ddddd"];;                                               
    - : (string * int) list = [("a", 1); ("bbb", 3); ("cc", 2); ("ddddd", 5)]
  \end{minted}

  \begin{solutionbox}{3cm}
    \begin{minted}{ocaml}
let rec stringy lst =
  match lst with
  | [] -> []
  | x::xs -> (x, String.length x)::stringy xs
    \end{minted}
  \end{solutionbox}


  \question 
  Write a function \texttt{positivesOf : int list -> int list}
  that returns the positive numbers in its input.
  \begin{minted}{ocaml}
    # positivesOf [-4; 9; 2; -8; -3; 1; 0];;
    - : int list = [9; 2; 1]
  \end{minted}
  
  \begin{solutionbox}{3.5cm}
    \begin{minted}{ocaml}
let rec positivesOf lst =
  match lst with
  | [] -> []
  | x::xs -> if x > 0 then x::positivesOf xs
             else positivesOf xs
    \end{minted}
  \end{solutionbox}

  
  \question 
  Write a function \texttt{gotcha : ('a -> bool) -> 'a list -> 'a} 
  that takes a predicate function \texttt{p} and a list \texttt{lst},
  and returns the first element \texttt{x} of \texttt{lst} for which 
  \texttt{p(x)} is true. If there is no such element,
  the function should fail with the error message 
  ``No soup for you!''.
  \begin{minted}{ocaml}
    # gotcha (fun n -> n > 5) [3; 4; 1; 2; 8; 4; 9; -8];;
    - : int = 8
    # gotcha (fun n -> n > 15) [3; 4; 1; 2; 8; 4; 9; -8];;
    Exception: Failure "No soup for you!".
  \end{minted}
  To make the program fail in the error case, use the 
  (\texttt{failwith "No soup for you!"}) expression.
  
  \begin{solutionbox}{3.5cm}
    \begin{minted}{ocaml}
let rec gotcha p lst =
  match lst with
  | [] -> failwith "No soup for you!"
  | x::xs -> if p x then x
             else gotcha p xs      
    \end{minted}
  \end{solutionbox}

  
  \question 
  Write a function \texttt{allUntil : ('a -> bool) -> 'a list -> 'a list} 
  that takes a predicate function \texttt{p}, a list \texttt{lst}, 
  and returns all the elements of \texttt{lst} up to the first element
  that does not satisfy \texttt{p}.
  \begin{minted}{ocaml}
    # allUntil (fun n -> n < 5) [3; 4; 1; 2; 8; 4; 9; -8];;
    - : int list = [3; 4; 1; 2]
    # allUntil (fun n -> n > 5) [3; 4; 1; 2; 8; 4; 9; -8];;
    - : int list = []
    # allUntil (fun n -> n < 15) [3; 4; 1; 2; 8; 4; 9; -8];;
    - : int list = [3; 4; 1; 2; 8; 4; 9; -8]
    # allUntil (fun s -> String.length(s) < 4) ["aa"; "bbb"; "c"; "dddd"; "eeeeeeee"; "fff"];;
    - : string list = ["aa"; "bbb"; "c"]
  \end{minted}
  
  \begin{solutionbox}{3.5cm}
    \begin{minted}{ocaml}
let rec allUntil p lst =
  match lst with
  | [] -> []
  | x::xs -> if p x then x::(allUntil p xs)
             else []
    \end{minted}
  \end{solutionbox}

  
  \question
  Write a function 
  \texttt{interleave : 'a list -> 'a list -> 'a list * 'a list} 
  that mixes its inputs by interleaving their elements.
  In this question, you may assume that the inputs will always have the same length;
  that is, I won't test your function with naughty inputs.
  \begin{minted}{ocaml}
    # interleave [1;2;3;4;5] [6;7;8;9;10];;                                   
    - : int list * int list = ([6; 2; 8; 4; 10], [1; 7; 3; 9; 5])
    # interleave [2;3;4;5] [7;8;9;10];;    
    - : int list * int list = ([7; 3; 9; 5], [2; 8; 4; 10])
  \end{minted}

  \begin{solutionbox}{4cm}
    \begin{minted}{ocaml}
let rec interleave lst1 lst2 =
  match lst1, lst2 with
  | ([], []) -> ([], [])
  | (x::xs, y::ys) ->
     let (left, right) = interleave xs ys
     in (y::right, x::left)
    \end{minted}
  \end{solutionbox}

  
  \question
  Write a function 
  \texttt{enumerate : 'a list -> ('a * int) list} 
  that enumerates the elements of its input with their index.
  The first element in a list is considered to be at index 0.
  You will want to write a helper function for this problem.
  \begin{minted}{ocaml}
    # enumerate ['a';'b';'c';'d';'e'];;
    - : (char * int) list = [('a',0);('b',1);('c',2);('d',3);('e',4)]
  \end{minted}

  \begin{solutionbox}{4cm}
    \begin{minted}{ocaml}
let enumerate lst =
  let rec aux lst index =
    match lst with
    | [] -> []
    | x::xs -> (x, index)::aux xs (index+1)
  in aux lst 0
    \end{minted}
  \end{solutionbox}

  
  \vspace{1em}
  \fbox{\parbox{0.9\textwidth}{
      \textbf{In all problems below, you must NOT use explicit recursion;
        use the library functions \texttt{map}, \texttt{fold\_left}, and \texttt{fold\_right}.}}}
  \vspace{1em}

  \question 
  Write a function \texttt{stringyWithMap}
  that is exactly the same as \texttt{stringy},
  but this time use \texttt{map}.

  \begin{solutionbox}{2cm}
    \begin{minted}{ocaml}
let stringyWithMap lst =
  List.map (fun s -> (s, String.length s)) lst
    \end{minted}
  \end{solutionbox}

  
  \question 
  Write a function \texttt{stringyWithFoldRight}
  that is exactly the same as \texttt{stringy},
  but this time use \texttt{fold\_right}.

  \begin{solutionbox}{2cm}
    \begin{minted}{ocaml}
let stringyWithFoldRight lst =
  List.fold_right (fun s acc -> (s, String.length s)::acc) lst []
    \end{minted}
  \end{solutionbox}

  
  \question 
  Write a function \texttt{stringyWithFoldLeft}
  that is exactly the same as \texttt{stringy},
  but this time use \texttt{fold\_left}.

  \begin{solutionbox}{2cm}
    \begin{minted}{ocaml}
let stringyWithFoldLeft lst =
  List.fold_left (fun acc s -> acc@[(s, String.length s)]) [] lst
    \end{minted}
  \end{solutionbox}

  
  \question 
  Write a function \texttt{positivesOfWithFoldRight}
  that is exactly the same as \texttt{positivesOf},
  but this time use \texttt{fold\_right}.

  \begin{solutionbox}{2cm}
    \begin{minted}{ocaml}
let positivesOfWithFoldRight lst =
  List.fold_right (fun x acc -> if x > 0 then x::acc else acc) lst []
    \end{minted}
  \end{solutionbox}

  
  \question 
  Write a function \texttt{positivesOfWithFoldLeft}
  that is exactly the same as \texttt{positivesOf},
  but this time use \texttt{fold\_left}.

  \begin{solutionbox}{2cm}
    \begin{minted}{ocaml}
let positivesOfWithFoldLeft lst =
  List.fold_left (fun acc x -> if x > 0 then acc@[x] else acc) [] lst
    \end{minted}
  \end{solutionbox}

  
  \question 
  Write a function \texttt{enumerateWithFoldLeft}
  that is exactly the same as \texttt{enumerate},
  but this time use \texttt{fold\_left}.

  \begin{solutionbox}{3.5cm}
    \begin{minted}{ocaml}
let enumerateWithFoldLeft lst =
  let f acc x =
    let (lst, index) = acc
    in (lst@[x,index], index+1)
  in fst(List.fold_left f ([], 0) lst)         
    \end{minted}
  \end{solutionbox}

  
  \vspace{1em}
  \fbox{\parbox{0.9\textwidth}{
      \textbf{In the problems below, you may use explicit recursion
        or the library functions such as \texttt{map}, \texttt{fold\_left},
        and \texttt{fold\_right}.
        It is a good idea to try solving the problems using both approaches.}}}
  \vspace{1em}

  \question 
  Implement the following functions:
  \texttt{rev}, \texttt{append}, \texttt{flatten},
  \texttt{map2}, \texttt{exists}, \texttt{mem},
  \texttt{partition}, \texttt{assoc}, \texttt{combine}.

  
  Their definitions are available in the \texttt{List} module:

  \url{http://caml.inria.fr/pub/docs/manual-ocaml/libref/List.html}


  \fbox{\textbf{In the problems below, your implementation is required to be tail-recursive.}}
  \vspace{1em}

  \question
  Write a function \texttt{positivesOf : int list -> int list}
  that returns the positive numbers in its input.
  \begin{minted}{fsharp}
    # positivesOf [-4; 9; 2; -8; -3; 1; 0];;
    - : int list = [9; 2; 1]
  \end{minted}

  \begin{solutionbox}{4cm}
    \begin{minted}{ocaml}
let positivesOf lst =
  let rec aux lst acc =
    match lst with
    | [] -> acc
    | x::xs -> aux xs (if x > 0 then x::acc else acc)
  in List.rev(aux lst [])
    \end{minted}
  \end{solutionbox}

  
  \question
  Write a function 
  \texttt{enumerate : 'a list -> ('a * int) list} 
  that enumerates the elements of its input with their index.
  The first element in a list is considered to be at index 0.
  \begin{minted}{fsharp}
    # enumerate ['a';'b';'c';'d';'e'];;
    - : (char * int) list = [('a',0);('b',1);('c',2);('d',3);('e',4)]
  \end{minted}
  \textbf{Extra exercise:} Solve the same problem when the elements are 
  enumerated from right to left. E.g:
  \begin{minted}{fsharp}
    # enumerate ['a';'b';'c';'d';'e'];;
    - : (char * int) list = [('a',4);('b',3);('c',2);('d',1);('e',0)]
  \end{minted}

  \begin{solutionbox}{4.5cm}
    \begin{minted}{ocaml}
let enumerate lst =
  let rec aux lst acc =
    match lst with
    | [] -> acc
    | x::xs -> let (eList, index) = acc
               in aux xs ((x, index)::eList, index+1)
  in List.rev(fst(aux lst ([], 0)))
    \end{minted}
  \end{solutionbox}

  
\end{questions}

\end{document}
