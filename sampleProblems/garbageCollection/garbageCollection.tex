\documentclass[addpoints]{exam}
\makeatletter
\expandafter\providecommand\expandafter*\csname ver@framed.sty\endcsname
{2003/07/21 v0.8a Simulated by exam}
\makeatother
\usepackage{framed}

%\printanswers

\usepackage{amsmath,bigstrut,minted,url,graphicx,tikz,marvosym,rotating}
\usepackage{ebproof}

\pagestyle{headandfoot}
\runningheadrule
\firstpageheader{}{}{Page \thepage\ of \numpages}
\runningheader{CS 321}{Sample Problems}{Page \thepage\ of \numpages}
\firstpagefooter{}{}{}
\runningfooter{}{}{}
              
\newcommand{\EJDG}[3]{#1 \vdash #2 \Rightarrow #3}
\newcommand{\TJDG}[3]{#1 \vdash #2 : #3}
\renewcommand{\L}[2]{\lambda #1.#2}

\begin{document}

\begin{center}
{\Large \textbf{
    Ozyegin University\\
    CS 321 Programming Languages\\
    Sample Problems on Garbage Collection
}}
\end{center}

\begin{questions}
  \question
  Which garbage collection algorithm would work the best for the following
  piece of Java program? Explain why. Remember that Strings in Java are immutable
  objects (i.e. once created, you may not modify them).
  
  \begin{minted}{java}
    String concat(String[] ss) {
      // Assume ss is very long
      String res = "";
      for (int i = 0; i < ss.length; i++)
          res += ss[i];
      return res;
    }
  \end{minted}

  Suggest a better (i.e. more efficient) way to implement this code.
  Explain why it would be more efficient.

  \begin{solutionbox}{4.5cm}
    This program creates a lot of short-lived objects.
    Each iteration of the loop creates a String object that becomes garbage
    in the next iteration.
    For this program, ``two space stop and copy'' probably works the best because
    this algorithm does not need to touch the many String objects that die young.
    Also, these young and dead objects may have caused fragmentation in the heap.
    ``Two space stop and copy'' fixes that problem as well.

    (Automatic) reference counting is likely to perform well, too.
    This is because there are no cyclic links between the dead objects.
    Furthermore, there are no links to other objects from the String objects;
    hence, deallocation of a String object does not cause a ripple effect. 
  \end{solutionbox}

  \question
  Write a piece of program (in Java or C++)
  that would cause a memory leak
  if reference counting were used
  as the garbage collection technique.

  \begin{solutionbox}{7.5cm}
    \begin{minted}{Java}
      class Node {
        int item;
        Node next;
      }
      ...
      Node node1 = new Node();
      Node node2 = new Node();
      node1.next = node2;
      node2.next = node1;
      node1 = null;
      node2 = null;
      ...
    \end{minted}
    The code above creates a cycle between two objects, and then loses the
    pointers to these objects from the stack. 
  \end{solutionbox}

  \newpage
  \question
  \strut
  \begin{parts}
    \part Suppose I have a program where I create a long linked list structure with no cycles,
    and then set my pointer to the head of the list to null,
    so the head becomes unreachable.
    Which garbage collection algorithm would perform the best for this case?
    Justify by explaining what drawbacks the other algorithms have for this case.
    \vfill

    \part Which garbage collection algorithm is likely to improve cache utilization? Why?
    \vfill

    \part Suppose I have a computer with small heap memory.
    If I am to make a choice between mark-and-sweep and two-space-stop-and-copy algorithms,
    which one should I choose? Why?
    \vfill

    \part Suppose I have a program where I create a lot of temporary (i.e. short-lived)
    objects of various sizes scattered (i.e. \emph{sa\c{c}{\i}lm{\i}\c{s}, yay{\i}lm\i\c{s}})
    all over the memory.
    If I am to make a choice between mark-and-sweep and two-space-stop-and-copy algorithms,
    which one should I choose? Why?
    \vfill

    \part Suppose I have a program where I create many long-lived and large objects and arrays.
    If I am to make a choice between mark-and-sweep and two-space-stop-and-copy algorithms,
    which one should I choose? Why?
    \vfill
  \end{parts}
  
  
\end{questions}
\end{document}
