\documentclass[addpoints]{exam}
\makeatletter
\expandafter\providecommand\expandafter*\csname ver@framed.sty\endcsname
{2003/07/21 v0.8a Simulated by exam}
\makeatother
\usepackage{framed}

%\printanswers

\usepackage{amsmath,bigstrut,minted,url,graphicx,tikz,marvosym,rotating}
\usepackage{ebproof}

\pagestyle{headandfoot}
\runningheadrule
\firstpageheader{}{}{Page \thepage\ of \numpages}
\runningheader{CS 321}{Sample Problems}{Page \thepage\ of \numpages}
\firstpagefooter{}{}{}
\runningfooter{}{}{}
              
\newcommand{\EJDG}[3]{#1 \vdash #2 \Rightarrow #3}
\newcommand{\TJDG}[3]{#1 \vdash #2 : #3}
\renewcommand{\L}[2]{\lambda #1.#2}

\begin{document}

\begin{center}
{\Large \textbf{
    Ozyegin University\\
    CS 321 Programming Languages\\
    Sample Problems on Streams
}}
\end{center}

Ee use the following definition of streams:

\begin{minted}{ocaml}
  type 'a stream = Cons of 'a * (unit -> 'a stream)
\end{minted}

\begin{questions}
  \question
  Define the stream of square numbers.
  That is, the stream of 1, 4, 9, 16, 25, 36, \ldots

  \begin{solutionbox}{3cm}
    {\footnotesize
    \begin{minted}{ocaml}
      let rec squaresFrom n = Cons(n*n, fun() -> squaresFrom (n+1))

      let squares = squaresFrom 1
    \end{minted}
    }
  \end{solutionbox}


  \question
  Implement a \texttt{cycle} function that takes a list
  \texttt{lst} and converts it to an infinite stream as if \texttt{lst}
  were a circular list.

  {\footnotesize
    \begin{minted}{ocaml}
      # cycle;;
      val cycle : 'a list -> 'a stream = <fun>
      # let letters = cycle ['a'; 'b'; 'c'; 'd'];;
      val letters : char stream = Cons ('a', <fun>)
      # take 15 letters;;
      - : char list =
      ['a'; 'b'; 'c'; 'd'; 'a'; 'b'; 'c'; 'd'; 'a'; 'b'; 'c'; 'd'; 'a'; 'b'; 'c']
      # take 10 (cycle ['a']);;
      - : char list = ['a'; 'a'; 'a'; 'a'; 'a'; 'a'; 'a'; 'a'; 'a'; 'a']  
    \end{minted}
    }
  \begin{solutionbox}{3cm}
    {\footnotesize
    \begin{minted}{ocaml}
        let cycle lst =                                                       
          let rec helper items = 
            match items with
            | [] -> helper lst
            | x::xs -> Cons(x, fun () -> helper xs)
          in helper lst
    \end{minted}
    }
  \end{solutionbox}
  
  \question
  Implement the \texttt{takeWhile} function
  for streams that
  takes a predicate \texttt{p}, a stream \texttt{s},
  and returns as a list all the elements of \texttt{s} 
  until there is an element of \texttt{s} that does not satisfy \texttt{p}.
  \textbf{Extra: give an efficient tail-recursive solution.}

  {\footnotesize
    \begin{minted}{ocaml}
# takeWhile (fun n -> n < 100) squares;;
- : int list = [1; 4; 9; 16; 25; 36; 49; 64; 81]
    \end{minted}
  }
  \begin{solutionbox}{3cm}
    {\footnotesize
    \begin{minted}{ocaml}
let takeWhile p st =
  let rec helper st acc =
    let x = head st
    in if p x then helper (tail st) (x::acc) else acc
  in List.rev(helper st [])
    \end{minted}
    }
  \end{solutionbox}

  \newpage
  \question
  Write an OCaml function \texttt{enumerate} that takes a
  stream and enumerates its elements starting from 0.
  
  {\footnotesize
    \begin{minted}{ocaml}
      # enumerate;;
      - : 'a stream -> (int * 'a) stream = <fun>
      # let letters = cycle ['e'; 'n'; 'u'; 'm'];;
      val letters : char stream = Cons ('a', <fun>)
      # take 10 letters;;                         
      - : char list = ['e'; 'n'; 'u'; 'm'; 'e'; 'n'; 'u'; 'm'; 'e'; 'n']
      # take 10 (enumerate letters);;
      - : (int * char) list =
      [(0, 'e'); (1, 'n'); (2, 'u'); (3, 'm'); (4, 'e'); (5, 'n'); (6, 'u');
        (7, 'm'); (8, 'e'); (9, 'n')]
    \end{minted}
  }
  \begin{solutionbox}{3cm}
    {\footnotesize
    \begin{minted}{ocaml}
let enumerate st =
  let rec helper n st =
    Cons((n, head st), fun() -> helper (n+1) (tail st))
  in helper 0 st
    \end{minted}
    }
  \end{solutionbox}

  \question
  Write an OCaml function named \texttt{merge} that combines two streams
  into one by taking elements alternately. (i.e. \emph{bi ordan bi burdan})
  
  {\footnotesize
    \begin{minted}{ocaml}
      # merge;;
      - : 'a stream -> 'a stream -> 'a stream = <fun>
      # let evens = filter (fun n -> n mod 2 = 0) naturals;;
      val evens : int stream = Cons (0, <fun>)
      # take 10 evens;;
      - : int list = [0; 2; 4; 6; 8; 10; 12; 14; 16; 18]
      # let odds = filter (fun n -> n mod 2 = 1) naturals;;
      val odds : int stream = Cons (1, <fun>)
      # take 10 odds;;                                     
      - : int list = [1; 3; 5; 7; 9; 11; 13; 15; 17; 19]
      # take 10 (merge odds evens);;
      - : int list = [1; 0; 3; 2; 5; 4; 7; 6; 9; 8]
      # take 10 (merge (cycle ['a']) (cycle ['b']));;
      - : char list = ['a'; 'b'; 'a'; 'b'; 'a'; 'b'; 'a'; 'b'; 'a'; 'b']
    \end{minted}
  }  
  \begin{solutionbox}{3cm}
    {\footnotesize
    \begin{minted}{ocaml}
let rec merge st1 st2 =
  Cons(head st1, fun() ->
                 Cons(head st2, fun() ->
                                merge (tail st1) (tail st2)))
    \end{minted}
    }
  \end{solutionbox}

\end{questions}
\end{document}
