\documentclass[addpoints]{exam}
\makeatletter
\expandafter\providecommand\expandafter*\csname ver@framed.sty\endcsname
{2003/07/21 v0.8a Simulated by exam}
\makeatother
\usepackage{framed}

%\printanswers

\usepackage{amsmath,bigstrut,minted,url,graphicx,tikz,marvosym,rotating}
\usepackage{ebproof}

\pagestyle{headandfoot}
\runningheadrule
\firstpageheader{}{}{Page \thepage\ of \numpages}
\runningheader{CS 321}{Sample Problems}{Page \thepage\ of \numpages}
\firstpagefooter{}{}{}
\runningfooter{}{}{}
              
\newcommand{\EJDG}[3]{#1 \vdash #2 \Rightarrow #3}
\newcommand{\TJDG}[3]{#1 \vdash #2 : #3}
\renewcommand{\L}[2]{\lambda #1.#2}

\begin{document}

\begin{center}
{\Large \textbf{
    Ozyegin University\\
    CS 321 Programming Languages\\
    Sample Problems on Imperative Programming
}}
\end{center}

\begin{questions}
  \question
  (PLC Ex. 7.2.(i))
  Write a C program containing a function
  \texttt{void arrsum(int n, int arr[], int *sump)}
  that computes and returns the sum of the first
  \texttt{n} elements of the given array \texttt{arr}.
  The result must be returned through the \texttt{sump} pointer.

  \begin{solutionbox}{4cm}
  \begin{minted}{c}
    void arrsum(int n, int arr[], int *sump) {
      int sum = 0;
      for (int i = 0; i < n; i++) {
        sum += arr[i];
      }
      *sump = sum;
    }
  \end{minted}
  \end{solutionbox}

  \question
  (PLC Ex. 7.2.(ii))
  Write a C program containing a function
  \texttt{void squares(int n, int arr[])} that,
  given \texttt{n} and an array \texttt{arr} of
  length \texttt{n} or more, fills \texttt{arr[i]}
  with \texttt{i*i} for \texttt{i} = $0,\ldots,n-1$.

  \begin{solutionbox}{3.5cm}
  \begin{minted}{c}
    void squares(int n, int arr[]) {
      for (int i = 0; i < n; i++) {
        arr[i] = i * i;
      }
    }
  \end{minted}
  \end{solutionbox}

  \question
  Write a \textbf{recursive} C function 
  \texttt{void fib(int n, int *res)}
  that computes the \texttt{n}$^{th}$ fibonacci number
  and returns it through the
  \texttt{res} pointer.

  \begin{solutionbox}{6cm}
  \begin{minted}{c}
    void fib(int n, int *res) {
      if (n == 0 || n == 1) {
        *res = 1;
      } else {
        int f1;
        fib(n-1, &f1);
        int f2;
        fib(n-2, &f2);
        *res = f1 + f2;
      }
    }
  \end{minted}
  \end{solutionbox}

  \newpage
  \question
  Assuming that the environment and the store are initially empty,
  give a possible environment and store at the
  end of the following piece of C program.
  \begin{minted}{c}
    int n = 38;
    int a[3] = {5, 9, 13};
    int *p;
    p = &a[1];
    *p = n;
    *(p+1) = a[0]*2;
    p = &n;
    p++;
  \end{minted}

  \begin{solution}
    After the third line, that is, after the declarations are handled, we may have
    and environment and a store such as below, where the value of \mintinline{c}{p} is
    garbage (i.e. it's not set, therefore it could be anything). 
    
    \begin{tabular}{|r c|}
      \multicolumn{2}{c}{Env}\\\hline
      n: & 70 \\
      a: & 74 \\
      p: & 75 \\
      \hline
    \end{tabular}
    \hspace{1cm}
    \begin{tabular}{c|c|c|c|c|c|c|c}
      \multicolumn{8}{c}{Store}\\
      \hline
      \multicolumn{1}{c}{} & \multicolumn{1}{c}{70} & \multicolumn{1}{c}{71} & \multicolumn{1}{c}{72} & \multicolumn{1}{c}{73} & \multicolumn{1}{c}{74} & \multicolumn{1}{c}{75} & \\\hline
      &  \bigstrut 38 &  5 &  9 & 13 & 71 & XX & \\[1ex]\hline
    \end{tabular}

    Executing the statement \mintinline{c}{p = &a[1];} gives us

    \begin{tabular}{|r c|}
      \hline
      n: & 70 \\
      a: & 74 \\
      p: & 75 \\
      \hline
    \end{tabular}
    \hspace{1cm}
    \begin{tabular}{c|c|c|c|c|c|c|c}
      \hline
      \multicolumn{1}{c}{} & \multicolumn{1}{c}{70} & \multicolumn{1}{c}{71} & \multicolumn{1}{c}{72} & \multicolumn{1}{c}{73} & \multicolumn{1}{c}{74} & \multicolumn{1}{c}{75} & \\\hline
      &  \bigstrut 38 &  5 &  9 & 13 & 71 & \textbf{72} & \\[1ex]\hline
    \end{tabular}
    
    Executing the statement \mintinline{c}{*p = n;} gives us

    \begin{tabular}{|r c|}
      \hline
      n: & 70 \\
      a: & 74 \\
      p: & 75 \\
      \hline
    \end{tabular}
    \hspace{1cm}
    \begin{tabular}{c|c|c|c|c|c|c|c}
      \hline
      \multicolumn{1}{c}{} & \multicolumn{1}{c}{70} & \multicolumn{1}{c}{71} & \multicolumn{1}{c}{72} & \multicolumn{1}{c}{73} & \multicolumn{1}{c}{74} & \multicolumn{1}{c}{75} & \\\hline
      &  \bigstrut 38 &  5 & \textbf{38} & 13 & 71 & 72 & \\[1ex]\hline
    \end{tabular}

    Executing the statement \mintinline{c}{*(p+1) = a[0]*2;} gives us

    \begin{tabular}{|r c|}
      \hline
      n: & 70 \\
      a: & 74 \\
      p: & 75 \\
      \hline
    \end{tabular}
    \hspace{1cm}
    \begin{tabular}{c|c|c|c|c|c|c|c}
      \hline
      \multicolumn{1}{c}{} & \multicolumn{1}{c}{70} & \multicolumn{1}{c}{71} & \multicolumn{1}{c}{72} & \multicolumn{1}{c}{73} & \multicolumn{1}{c}{74} & \multicolumn{1}{c}{75} & \\\hline
      &  \bigstrut 38 &  5 & 38 & \textbf{10} & 71 & 72 & \\[1ex]\hline
    \end{tabular}

    Executing the statement \mintinline{c}{p = &n;} gives us

    \begin{tabular}{|r c|}
      \hline
      n: & 70 \\
      a: & 74 \\
      p: & 75 \\
      \hline
    \end{tabular}
    \hspace{1cm}
    \begin{tabular}{c|c|c|c|c|c|c|c}
      \hline
      \multicolumn{1}{c}{} & \multicolumn{1}{c}{70} & \multicolumn{1}{c}{71} & \multicolumn{1}{c}{72} & \multicolumn{1}{c}{73} & \multicolumn{1}{c}{74} & \multicolumn{1}{c}{75} & \\\hline
      &  \bigstrut 38 &  5 & 38 & 10 & 71 & \textbf{70} & \\[1ex]\hline
    \end{tabular}

    Executing the statement \mintinline{c}{p++;} gives us

    \begin{tabular}{|r c|}
      \hline
      n: & 70 \\
      a: & 74 \\
      p: & 75 \\
      \hline
    \end{tabular}
    \hspace{1cm}
    \begin{tabular}{c|c|c|c|c|c|c|c}
      \hline
      \multicolumn{1}{c}{} & \multicolumn{1}{c}{70} & \multicolumn{1}{c}{71} & \multicolumn{1}{c}{72} & \multicolumn{1}{c}{73} & \multicolumn{1}{c}{74} & \multicolumn{1}{c}{75} & \\\hline
      &  \bigstrut 38 &  5 & 38 & 10 & 71 & \textbf{71} & \\[1ex]\hline
    \end{tabular}
  \end{solution}

  \newpage
  \question
  In C++, parameters of functions that are declared using the 
  \& operator are passed by reference.
  What is the output of the C++-like program below?
  \begin{minted}{cpp}
  void main() {
    int a[3] = {3, 7, 10};
    int m = 56;
    int n = 99;
    mystery(m, n, a);
    print m, n, a[0], a[1], a[2]
  }
  void mystery(int x, int &y, int a[]) {
    a[0] += 1;
    int temp = x;
    x = y;
    y = temp;
  }
  \end{minted}

  \begin{solutionbox}{1cm}
  56 56 4 7 10
  \end{solutionbox}

  \question
  An array can be represented as a pointer.
  For instance, the array definition
  \begin{verbatim}
     int a[4] = {12, 13, 14, 15}; \end{verbatim}
  can be represented using the following env/store:
{\small
  \begin{verbatim}
     Env =>      a: 54

     Store =>    --------------------------------------
                     | 12 | 13 | 14 | 15 | 50 |    |    
                 --------------------------------------
                       50   51   52   53   54   55  
  \end{verbatim}}   
  Explain how you can use this representation to find the length
  of an array.

  \begin{solutionbox}{1.5cm}
  Find the difference between the address and the value
  of the array variable. That is,\\
  \texttt{\&a - a}
  \end{solutionbox}

  \question
  A C-like program is given below.
  \begin{verbatim}
m = &n;         // A
*m = k[2];      // B
k--; m++;       // C
*k = *m;        // D
m[2] = n;       // E
  \end{verbatim}
  Starting from the env. and store given below,
  show the environment and the store 
  after \textbf{each} statement.
{\small
\begin{verbatim}
Env =>      m: 50, n: 51, k:56

Store =>    -----------------------------------------
               | 48 | 53 | 49 | 17 | 41 | 50 | 52 |  
            -----------------------------------------
                 50   51   52   53   54   55   56   
\end{verbatim}}


  \begin{solution}
After A:
{\small
\begin{verbatim}
Env =>      m: 50, n: 51, k:56

Store =>    -----------------------------------------
               | 51 | 53 | 49 | 17 | 41 | 50 | 52 |  
            -----------------------------------------
                 50   51   52   53   54   55   56   
\end{verbatim}}

After B:
{\small
\begin{verbatim}
Env =>      m: 50, n: 51, k:56

Store =>    -----------------------------------------
               | 51 | 41 | 49 | 17 | 41 | 50 | 52 |  
            -----------------------------------------
                 50   51   52   53   54   55   56   
\end{verbatim}}

After C:
{\small
\begin{verbatim}
Env =>      m: 50, n: 51, k:56

Store =>    -----------------------------------------
               | 52 | 41 | 49 | 17 | 41 | 50 | 51 |  
            -----------------------------------------
                 50   51   52   53   54   55   56   
\end{verbatim}}

After D:
{\small
\begin{verbatim}
Env =>      m: 50, n: 51, k:56

Store =>    -----------------------------------------
               | 52 | 49 | 49 | 17 | 41 | 50 | 51 |  
            -----------------------------------------
                 50   51   52   53   54   55   56   
\end{verbatim}}

After E:
{\small
\begin{verbatim}
Env =>      m: 50, n: 51, k:56

Store =>    -----------------------------------------
               | 52 | 49 | 49 | 17 | 49 | 50 | 51 |  
            -----------------------------------------
                 50   51   52   53   54   55   56   
\end{verbatim}}
  \end{solution}

  \newpage
  \question
  In C++, parameters of functions that are declared using the \& operator are passed by reference.
  A C++ program is given below.

  {\small
    \begin{minipage}[t]{0.30\textwidth}
    \begin{minted}{c++}
    void main() {
      int a[3] = {8, 15, 6};
      int b = 88;
      int c = 33;
      // A
      f(b, &c, a);
      // I
    }
    \end{minted}
    \end{minipage}
    \begin{minipage}[t]{0.40\textwidth}
    \begin{minted}{c++}
    void f(int &x, int *y, int z[]) {
      // B
      *y = *y + 1;
      // C
      y = &(z[1]);
      // D
      y++; x++;
      // E
      *y = x + 10;
      // F
      x = *y;
      // G
      z[2] = -1;
      // H
    }
    \end{minted}
    \end{minipage}
  }

  The \emph{environment} maps names to locations;
  the \emph{store} maps locations to values. 
  Suppose the environment and the store at point A are as given below.
  Give possible environment and store configurations for points B--I.
  It is OK to show the changes only.
  
  \texttt{A}:  
  \begin{tabular}{|r c|}
    \multicolumn{2}{c}{Env}\\\hline
    a: & 53 \\
    b: & 54 \\
    c: & 55 \\
    \hline
  \end{tabular}
  \hspace{1cm}
  \begin{tabular}{c|c|c|c|c|c|c|c|c|c|c|c|c|c}
    \multicolumn{11}{c}{Store}\\
    \hline
    \multicolumn{1}{c}{} & \multicolumn{1}{c}{50} & \multicolumn{1}{c}{51} & \multicolumn{1}{c}{52}
    & \multicolumn{1}{c}{53} & \multicolumn{1}{c}{54} & \multicolumn{1}{c}{55}
    & \multicolumn{1}{c}{56} & \multicolumn{1}{c}{57} & \multicolumn{1}{c}{58}
    & \multicolumn{1}{c}{59} & \multicolumn{1}{c}{60} & \multicolumn{1}{c}{61} & \\\hline
    &  \bigstrut 8 &  15 &  6 & 50 & 88 & 33 &  &  &  &  &  &  \\[1ex]\hline
  \end{tabular}
  \hfill
  \strut
  \vspace{0.1cm}
  
  \texttt{B}:  
  \begin{tabular}{|r c|}
    \hline
    x: & \ifprintanswers 54 \else \strut\hspace{0.4cm} \fi\\
    y: & \ifprintanswers 56 \else \strut\hspace{0.4cm} \fi\\
    z: & \ifprintanswers 57 \else \strut\hspace{0.4cm} \fi\\
    \hline
  \end{tabular}
  \hspace{1cm}
  \begin{tabular}{c|c|c|c|c|c|c|c|c|c|c|c|c|c}
    \hline
    \multicolumn{1}{c}{} & \multicolumn{1}{c}{50} & \multicolumn{1}{c}{51} & \multicolumn{1}{c}{52}
    & \multicolumn{1}{c}{53} & \multicolumn{1}{c}{54} & \multicolumn{1}{c}{55}
    & \multicolumn{1}{c}{56} & \multicolumn{1}{c}{57} & \multicolumn{1}{c}{58}
    & \multicolumn{1}{c}{59} & \multicolumn{1}{c}{60} & \multicolumn{1}{c}{61} & \\\hline
    &  \bigstrut   &    &    &   &   &  & \ifprintanswers 55 \fi & \ifprintanswers 50 \fi &   &  &  & \\[1ex]\hline
  \end{tabular}
  \hfill
  \strut
  \vspace{0.1cm}
  
  \texttt{C}:  
  \begin{tabular}{|r c|}
    \hline
    x: & \strut\hspace{0.4cm} \\
    y: &  \\
    z: &  \\
    \hline
  \end{tabular}
  \hspace{1cm}
  \begin{tabular}{c|c|c|c|c|c|c|c|c|c|c|c|c|c}
    \hline
    \multicolumn{1}{c}{} & \multicolumn{1}{c}{50} & \multicolumn{1}{c}{51} & \multicolumn{1}{c}{52}
    & \multicolumn{1}{c}{53} & \multicolumn{1}{c}{54} & \multicolumn{1}{c}{55}
    & \multicolumn{1}{c}{56} & \multicolumn{1}{c}{57} & \multicolumn{1}{c}{58}
    & \multicolumn{1}{c}{59} & \multicolumn{1}{c}{60} & \multicolumn{1}{c}{61} & \\\hline
    &  \bigstrut   &    &    &   &   & \ifprintanswers 34 \fi  &  &   &   &  &  & \\[1ex]\hline
  \end{tabular}
  \hfill
  \strut
  \vspace{0.1cm}
  
  \texttt{D}:  
  \begin{tabular}{|r c|}
    \hline
    x: & \strut\hspace{0.4cm} \\
    y: &  \\
    z: &  \\
    \hline
  \end{tabular}
  \hspace{1cm}
  \begin{tabular}{c|c|c|c|c|c|c|c|c|c|c|c|c|c}
    \hline
    \multicolumn{1}{c}{} & \multicolumn{1}{c}{50} & \multicolumn{1}{c}{51} & \multicolumn{1}{c}{52}
    & \multicolumn{1}{c}{53} & \multicolumn{1}{c}{54} & \multicolumn{1}{c}{55}
    & \multicolumn{1}{c}{56} & \multicolumn{1}{c}{57} & \multicolumn{1}{c}{58}
    & \multicolumn{1}{c}{59} & \multicolumn{1}{c}{60} & \multicolumn{1}{c}{61} & \\\hline
    &  \bigstrut   &    &    &   &   &  & \ifprintanswers 51 \fi &  &   &  &  & \\[1ex]\hline
  \end{tabular}
  \hfill
  \strut
  \vspace{0.1cm}
  
  \texttt{E}:  
  \begin{tabular}{|r c|}
    \hline
    x: & \strut\hspace{0.4cm} \\
    y: &  \\
    z: &  \\
    \hline
  \end{tabular}
  \hspace{1cm}
  \begin{tabular}{c|c|c|c|c|c|c|c|c|c|c|c|c|c}
    \hline
    \multicolumn{1}{c}{} & \multicolumn{1}{c}{50} & \multicolumn{1}{c}{51} & \multicolumn{1}{c}{52}
    & \multicolumn{1}{c}{53} & \multicolumn{1}{c}{54} & \multicolumn{1}{c}{55}
    & \multicolumn{1}{c}{56} & \multicolumn{1}{c}{57} & \multicolumn{1}{c}{58}
    & \multicolumn{1}{c}{59} & \multicolumn{1}{c}{60} & \multicolumn{1}{c}{61} & \\\hline
    &  \bigstrut   &    &    &   & \ifprintanswers 89 \fi &  & \ifprintanswers 52 \fi &  &   &  &  & \\[1ex]\hline
  \end{tabular}
  \hfill
  \strut
  \vspace{0.1cm}
  
  \texttt{F}:  
  \begin{tabular}{|r c|}
    \hline
    x: & \strut\hspace{0.4cm} \\
    y: &  \\
    z: &  \\
    \hline
  \end{tabular}
  \hspace{1cm}
  \begin{tabular}{c|c|c|c|c|c|c|c|c|c|c|c|c|c}
    \hline
    \multicolumn{1}{c}{} & \multicolumn{1}{c}{50} & \multicolumn{1}{c}{51} & \multicolumn{1}{c}{52}
    & \multicolumn{1}{c}{53} & \multicolumn{1}{c}{54} & \multicolumn{1}{c}{55}
    & \multicolumn{1}{c}{56} & \multicolumn{1}{c}{57} & \multicolumn{1}{c}{58}
    & \multicolumn{1}{c}{59} & \multicolumn{1}{c}{60} & \multicolumn{1}{c}{61} & \\\hline
    &  \bigstrut   &    & \ifprintanswers 99 \fi &   &   &   &  &   &   &  &  & \\[1ex]\hline
  \end{tabular}
  \hfill
  \strut
  \vspace{0.1cm}
  
  \texttt{G}:  
  \begin{tabular}{|r c|}
    \hline
    x: & \strut\hspace{0.4cm} \\
    y: &  \\
    z: &  \\
    \hline
  \end{tabular}
  \hspace{1cm}
  \begin{tabular}{c|c|c|c|c|c|c|c|c|c|c|c|c|c}
    \hline
    \multicolumn{1}{c}{} & \multicolumn{1}{c}{50} & \multicolumn{1}{c}{51} & \multicolumn{1}{c}{52}
    & \multicolumn{1}{c}{53} & \multicolumn{1}{c}{54} & \multicolumn{1}{c}{55}
    & \multicolumn{1}{c}{56} & \multicolumn{1}{c}{57} & \multicolumn{1}{c}{58}
    & \multicolumn{1}{c}{59} & \multicolumn{1}{c}{60} & \multicolumn{1}{c}{61} & \\\hline
    &  \bigstrut   &    &    &   & \ifprintanswers 99 \fi &   &  &   &  &  &  & \\[1ex]\hline
  \end{tabular}
  \hfill
  \strut
  \vspace{0.1cm}
  
  \texttt{H}:  
  \begin{tabular}{|r c|}
    \hline
    x: & \strut\hspace{0.4cm} \\
    y: &  \\
    z: &  \\
    \hline
  \end{tabular}
  \hspace{1cm}
  \begin{tabular}{c|c|c|c|c|c|c|c|c|c|c|c|c|c}
    \hline
    \multicolumn{1}{c}{} & \multicolumn{1}{c}{50} & \multicolumn{1}{c}{51} & \multicolumn{1}{c}{52}
    & \multicolumn{1}{c}{53} & \multicolumn{1}{c}{54} & \multicolumn{1}{c}{55}
    & \multicolumn{1}{c}{56} & \multicolumn{1}{c}{57} & \multicolumn{1}{c}{58}
    & \multicolumn{1}{c}{59} & \multicolumn{1}{c}{60} & \multicolumn{1}{c}{61} & \\\hline
    &  \bigstrut   &    & \ifprintanswers -1 \fi &   &   &   &  &   &   &  &  & \\[1ex]\hline
  \end{tabular}
  \hfill
  \strut
  \vspace{0.1cm}
  
  \texttt{I}:
  \begin{tabular}{|r c|}
    \hline
    a: & \ifprintanswers 53 \else \strut\hspace{0.4cm} \fi\\
    b: & \ifprintanswers 54 \else \strut\hspace{0.4cm} \fi\\
    c: & \ifprintanswers 55 \else \strut\hspace{0.4cm} \fi\\
    \hline
  \end{tabular}
  \hspace{1cm}
  \begin{tabular}{c|c|c|c|c|c|c|c|c|c|c|c|c|c}
    \hline
    \multicolumn{1}{c}{} & \multicolumn{1}{c}{50} & \multicolumn{1}{c}{51} & \multicolumn{1}{c}{52}
    & \multicolumn{1}{c}{53} & \multicolumn{1}{c}{54} & \multicolumn{1}{c}{55}
    & \multicolumn{1}{c}{56} & \multicolumn{1}{c}{57} & \multicolumn{1}{c}{58}
    & \multicolumn{1}{c}{59} & \multicolumn{1}{c}{60} & \multicolumn{1}{c}{61} & \\\hline
    &  \bigstrut   &    &    &   &   &   &  &   &   &  &  & \\[1ex]\hline
  \end{tabular}
  \hfill
  \strut


\end{questions}
\end{document}
